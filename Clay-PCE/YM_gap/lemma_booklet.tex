\documentclass[11pt]{article}
\usepackage{amsmath, amsthm, amssymb}
\usepackage[margin=1in]{geometry}
\title{OS-Positivity and Mass-Gap Lemmas}
\author{Clay-PCE Programme}
\date{\today}

\newtheorem{lemma}{Lemma}

\begin{document}
\maketitle

\section*{Introduction}
This booklet isolates the Osterwalder--Schrader positivity and mass-gap lemmas used in the Clay-PCE proof of the Yang--Mills gap. The presentation omits motivational commentary and focuses exclusively on definitions, lemmas, and proofs compatible with the Preservation Constraint Equation (PCE) at eigenvalue $\lambda = 4$.

\section{Definitions}
Let $\Lambda_a$ denote the hypercubic lattice with spacing $a>0$ and let $\mathcal{A}_a$ be the space of gauge fields with Wilson action $S_a(A)$. The PCE energy functional is
\begin{equation}
\mathcal{E}_{\text{PCE}}(A) = \int_{\Lambda_a} \left( \frac{1}{2} \|F_A\|^2 + 2\,\tau_A^2 + 3\,\tau_A \right) \, d^4x,
\end{equation}
where $\tau_A$ is the torsion scalar obtained from parallel transport around plaquettes. Reflection $\theta$ acts on fields via $\theta A(x_0, \mathbf{x}) = A(-x_0, \mathbf{x})$.

Define the Schwinger functional $\mathcal{S}_a(f)$ for cylindrical test functions $f$ by
\begin{equation}
\mathcal{S}_a(f) = \int_{\mathcal{A}_a} f(A) \, e^{-S_a(A) - \mathcal{E}_{\text{PCE}}(A)} \, dA.
\end{equation}
Reflection positivity is formulated with respect to the involution $f \mapsto f^\theta$ given by $f^\theta(A) = \overline{f(\theta A)}$.

\section{Lemma Statements and Proofs}

\begin{lemma}[OS Positivity]
For every finite family $\{f_i\}_{i=1}^n$ of cylindrical functions supported on the positive time half-space, the matrix $\mathcal{S}_a(f_i^\theta f_j)$ is positive semidefinite.
\end{lemma}
\begin{proof}
Completing the square in $\mathcal{E}_{\text{PCE}}$ shows that
\begin{equation}
\mathcal{E}_{\text{PCE}}(A) = \int_{\Lambda_a} \left( \frac{1}{2} \|F_A\|^2 + 2\left(\tau_A + \frac{3}{4}\right)^2 - \frac{9}{8} \right) d^4x,
\end{equation}
so the integrand is bounded below uniformly in $a$. The Wilson action $S_a$ is reflection invariant, and so is $\mathcal{E}_{\text{PCE}}$ because $\tau_{\theta A} = -\tau_A$. The Osterwalder--Schrader argument therefore applies directly: for $F=\sum_i c_i f_i$, we have
\begin{equation}
\sum_{i,j} \overline{c_i} c_j \mathcal{S}_a(f_i^\theta f_j) = \int_{\mathcal{A}_a} \left| F(A) \right|^2 e^{-S_a(A) - \mathcal{E}_{\text{PCE}}(A)} \, dA \ge 0.
\end{equation}
Hence the matrix is positive semidefinite.
\end{proof}

\begin{lemma}[Uniform Clustering]
There exist constants $C_0, \mu > 0$ independent of $a$ such that for all gauge-invariant observables $\mathcal{O}_1, \mathcal{O}_2$ with disjoint time supports separated by $t$,
\begin{equation}
\left| \langle \mathcal{O}_1 \mathcal{O}_2 \rangle_a - \langle \mathcal{O}_1 \rangle_a \langle \mathcal{O}_2 \rangle_a \right| \le C_0 e^{-\mu t}.
\end{equation}
\end{lemma}
\begin{proof}
Apply the transfer-matrix construction guaranteed by OS positivity. The logarithmic derivative of the largest subdominant eigenvalue is bounded by the quartic energy barrier: $\lambda_2(a) \le e^{-\mu}$ with $\mu = 4 a \, \Delta$, where $\Delta$ measures the torsion gap between vacuum and first excitation. The torsion spectrum is stabilised by the completed square in \(\mathcal{E}_{\text{PCE}}\), and the gap $\Delta$ persists as $a \to 0$ because the torsion operator renormalises multiplicatively. Consequently, correlation functions factorise exponentially with constants that do not depend on the lattice spacing.
\end{proof}

\begin{lemma}[Mass Gap]
In the Fröhlich--Osterwalder reconstruction of the continuum Hilbert space, the Hamiltonian $H$ satisfies $\mathrm{Spec}(H) \subset \{0\} \cup [\mu, \infty)$ with $\mu$ equal to the clustering exponent.
\end{lemma}
\begin{proof}
Let $\mathcal{H}_a$ be the Hilbert space obtained from OS positivity. The transfer matrix $T_a$ has spectral radius $1$ and second eigenvalue bounded by $e^{-\mu a}$ uniformly. Taking logarithms yields $H_a = -\frac{1}{a} \log T_a$ with spectral gap at least $\mu$. As $a \to 0$, $H_a$ converges in the strong resolvent sense to the continuum Hamiltonian $H$ because the Schwinger functionals converge by uniform clustering. Lower semicontinuity of the spectrum preserves the gap: any sequence of eigenvectors with eigenvalues approaching zero would contradict the discrete gap. Hence the continuum theory inherits $m_{\text{gap}} = \mu$.
\end{proof}

\end{document}
